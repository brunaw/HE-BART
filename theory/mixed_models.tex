\PassOptionsToPackage{unicode=true}{hyperref} % options for packages loaded elsewhere
\PassOptionsToPackage{hyphens}{url}
\PassOptionsToPackage{dvipsnames,svgnames*,x11names*}{xcolor}
%
\documentclass[]{article}
\usepackage{lmodern}
\usepackage{amssymb,amsmath}
\usepackage{ifxetex,ifluatex}
\usepackage{fixltx2e} % provides \textsubscript
\ifnum 0\ifxetex 1\fi\ifluatex 1\fi=0 % if pdftex
  \usepackage[T1]{fontenc}
  \usepackage[utf8]{inputenc}
  \usepackage{textcomp} % provides euro and other symbols
\else % if luatex or xelatex
  \usepackage{unicode-math}
  \defaultfontfeatures{Ligatures=TeX,Scale=MatchLowercase}
\fi
% use upquote if available, for straight quotes in verbatim environments
\IfFileExists{upquote.sty}{\usepackage{upquote}}{}
% use microtype if available
\IfFileExists{microtype.sty}{%
\usepackage[]{microtype}
\UseMicrotypeSet[protrusion]{basicmath} % disable protrusion for tt fonts
}{}
\IfFileExists{parskip.sty}{%
\usepackage{parskip}
}{% else
\setlength{\parindent}{0pt}
\setlength{\parskip}{6pt plus 2pt minus 1pt}
}
\usepackage{xcolor}
\usepackage{hyperref}
\hypersetup{
            colorlinks=true,
            linkcolor=Maroon,
            filecolor=Maroon,
            citecolor=Blue,
            urlcolor=blue,
            breaklinks=true}
\urlstyle{same}  % don't use monospace font for urls
\usepackage[margin=1in]{geometry}
\usepackage{color}
\usepackage{fancyvrb}
\newcommand{\VerbBar}{|}
\newcommand{\VERB}{\Verb[commandchars=\\\{\}]}
\DefineVerbatimEnvironment{Highlighting}{Verbatim}{commandchars=\\\{\}}
% Add ',fontsize=\small' for more characters per line
\usepackage{framed}
\definecolor{shadecolor}{RGB}{248,248,248}
\newenvironment{Shaded}{\begin{snugshade}}{\end{snugshade}}
\newcommand{\AlertTok}[1]{\textcolor[rgb]{0.94,0.16,0.16}{#1}}
\newcommand{\AnnotationTok}[1]{\textcolor[rgb]{0.56,0.35,0.01}{\textbf{\textit{#1}}}}
\newcommand{\AttributeTok}[1]{\textcolor[rgb]{0.77,0.63,0.00}{#1}}
\newcommand{\BaseNTok}[1]{\textcolor[rgb]{0.00,0.00,0.81}{#1}}
\newcommand{\BuiltInTok}[1]{#1}
\newcommand{\CharTok}[1]{\textcolor[rgb]{0.31,0.60,0.02}{#1}}
\newcommand{\CommentTok}[1]{\textcolor[rgb]{0.56,0.35,0.01}{\textit{#1}}}
\newcommand{\CommentVarTok}[1]{\textcolor[rgb]{0.56,0.35,0.01}{\textbf{\textit{#1}}}}
\newcommand{\ConstantTok}[1]{\textcolor[rgb]{0.00,0.00,0.00}{#1}}
\newcommand{\ControlFlowTok}[1]{\textcolor[rgb]{0.13,0.29,0.53}{\textbf{#1}}}
\newcommand{\DataTypeTok}[1]{\textcolor[rgb]{0.13,0.29,0.53}{#1}}
\newcommand{\DecValTok}[1]{\textcolor[rgb]{0.00,0.00,0.81}{#1}}
\newcommand{\DocumentationTok}[1]{\textcolor[rgb]{0.56,0.35,0.01}{\textbf{\textit{#1}}}}
\newcommand{\ErrorTok}[1]{\textcolor[rgb]{0.64,0.00,0.00}{\textbf{#1}}}
\newcommand{\ExtensionTok}[1]{#1}
\newcommand{\FloatTok}[1]{\textcolor[rgb]{0.00,0.00,0.81}{#1}}
\newcommand{\FunctionTok}[1]{\textcolor[rgb]{0.00,0.00,0.00}{#1}}
\newcommand{\ImportTok}[1]{#1}
\newcommand{\InformationTok}[1]{\textcolor[rgb]{0.56,0.35,0.01}{\textbf{\textit{#1}}}}
\newcommand{\KeywordTok}[1]{\textcolor[rgb]{0.13,0.29,0.53}{\textbf{#1}}}
\newcommand{\NormalTok}[1]{#1}
\newcommand{\OperatorTok}[1]{\textcolor[rgb]{0.81,0.36,0.00}{\textbf{#1}}}
\newcommand{\OtherTok}[1]{\textcolor[rgb]{0.56,0.35,0.01}{#1}}
\newcommand{\PreprocessorTok}[1]{\textcolor[rgb]{0.56,0.35,0.01}{\textit{#1}}}
\newcommand{\RegionMarkerTok}[1]{#1}
\newcommand{\SpecialCharTok}[1]{\textcolor[rgb]{0.00,0.00,0.00}{#1}}
\newcommand{\SpecialStringTok}[1]{\textcolor[rgb]{0.31,0.60,0.02}{#1}}
\newcommand{\StringTok}[1]{\textcolor[rgb]{0.31,0.60,0.02}{#1}}
\newcommand{\VariableTok}[1]{\textcolor[rgb]{0.00,0.00,0.00}{#1}}
\newcommand{\VerbatimStringTok}[1]{\textcolor[rgb]{0.31,0.60,0.02}{#1}}
\newcommand{\WarningTok}[1]{\textcolor[rgb]{0.56,0.35,0.01}{\textbf{\textit{#1}}}}
\usepackage{graphicx,grffile}
\makeatletter
\def\maxwidth{\ifdim\Gin@nat@width>\linewidth\linewidth\else\Gin@nat@width\fi}
\def\maxheight{\ifdim\Gin@nat@height>\textheight\textheight\else\Gin@nat@height\fi}
\makeatother
% Scale images if necessary, so that they will not overflow the page
% margins by default, and it is still possible to overwrite the defaults
% using explicit options in \includegraphics[width, height, ...]{}
\setkeys{Gin}{width=\maxwidth,height=\maxheight,keepaspectratio}
\setlength{\emergencystretch}{3em}  % prevent overfull lines
\providecommand{\tightlist}{%
  \setlength{\itemsep}{0pt}\setlength{\parskip}{0pt}}
\setcounter{secnumdepth}{5}
% Redefines (sub)paragraphs to behave more like sections
\ifx\paragraph\undefined\else
\let\oldparagraph\paragraph
\renewcommand{\paragraph}[1]{\oldparagraph{#1}\mbox{}}
\fi
\ifx\subparagraph\undefined\else
\let\oldsubparagraph\subparagraph
\renewcommand{\subparagraph}[1]{\oldsubparagraph{#1}\mbox{}}
\fi

% set default figure placement to htbp
\makeatletter
\def\fps@figure{htbp}
\makeatother

\usepackage{amsmath}
\usepackage{mathtools}
\usepackage{float}
\mathtoolsset{showonlyrefs}
\floatplacement{figure}{H}
\newcommand{\horrule}[1]{\rule{\linewidth}{#1}}

\author{}
\date{\vspace{-2.5em}}

\begin{document}

\title{  
 \normalfont \normalsize 
 \textsc{Mixed Models Theory}\\[25pt]
\author{Bruna Wundervald}
\date{\normalsize August, 2020}
\rule{\linewidth}{2pt} \\[ .5cm]}

\maketitle

\vspace{\fill}

\tableofcontents

\rule{\linewidth}{1pt}

\newpage

\hypertarget{linear-mixed-models}{%
\section{Linear Mixed Models}\label{linear-mixed-models}}

\hypertarget{model-statement}{%
\subsection{Model Statement}\label{model-statement}}

A LMM for a set of observations \(y = (y_1, \dots, y_n)\) has the
general form

\begin{equation}
Y | b \sim N(\mu, \Sigma), \quad \mu = X\beta + Zb, \quad b \sim N(0, \Sigma_{b}),
\end{equation}

where \(X\) and \(Z\) are the \(p \times n\) predictor matrices, and
\(\Sigma = \sigma^2 I\) usually. An example for clustered data is:

\begin{equation}
Y_{ij} \sim N(\mu_{ij}, \Sigma), \quad \mu_{ij} = x^{T}_{ij} \beta + z^{T}_{ij} b, \quad b_i \sim N(0, \Sigma^{*}_{b}),
\end{equation}

where \(x_{ij}\) now contains the predictor values for the \(j\)-th
observation in the \(i-\)th cluster, and \(z_{ij}\) is the sub-vector of
\(x_{ij}\) that exhibits extra between cluster variation in its
relationship to \(Y\).

\hypertarget{example-from-the-lme4-paper}{%
\subsubsection{\texorpdfstring{Example from the \texttt{lme4}
paper}{Example from the lme4 paper}}\label{example-from-the-lme4-paper}}

Let us consider now the data from a sleep deprivation study, from the
\texttt{lme4} package paper. On day 0 the subjects had their normal
amount of sleep, and were from that night restricted to 3 hours of sleep
per night. The response variable represents the average reaction times
in milliseconds (ms) on a series of tests done each day for each
subject. In the following figure, we can see a general trend of the
reaction time increasing with the passage of days, and the reaction time
itself varies quite a lot between the subjects, in both slope (starting
reaction time) and intercepts (effect of days in the reaction time).
This type of study justifies the use of a LMM, as we can clear see that
there is a difference in the response between individuals and days.

\begin{figure}

{\centering \includegraphics{mixed_models_files/figure-latex/unnamed-chunk-1-1} 

}

\caption{Reaction time per days and subjects in the sleep deprivation study.}\label{fig:unnamed-chunk-1}
\end{figure}

The following code starts by creating an LMM model using the subject ID
as the random effect, for which the standard error is estimated as 37.12
(26.01, 52.94). The second model now adds the Days variable a a random
effect slope, meaning the random intercept and slope are correlated
(-0.48, 0.68). In comparison to the previous model, the ICs for the
fixed effects got smaller, given that now we have less uncertainty about
the population average.

\begin{Shaded}
\begin{Highlighting}[]
\NormalTok{lmm <-}\StringTok{ }\KeywordTok{lmer}\NormalTok{(Reaction }\OperatorTok{~}\StringTok{ }\NormalTok{Days }\OperatorTok{+}\StringTok{ }\NormalTok{(}\DecValTok{1} \OperatorTok{|}\StringTok{ }\NormalTok{Subject), sleepstudy)}
\KeywordTok{summary}\NormalTok{(lmm)}
\end{Highlighting}
\end{Shaded}

\begin{verbatim}
## Linear mixed model fit by REML ['lmerMod']
## Formula: Reaction ~ Days + (1 | Subject)
##    Data: sleepstudy
## 
## REML criterion at convergence: 1786.5
## 
## Scaled residuals: 
##     Min      1Q  Median      3Q     Max 
## -3.2257 -0.5529  0.0109  0.5188  4.2506 
## 
## Random effects:
##  Groups   Name        Variance Std.Dev.
##  Subject  (Intercept) 1378.2   37.12   
##  Residual              960.5   30.99   
## Number of obs: 180, groups:  Subject, 18
## 
## Fixed effects:
##             Estimate Std. Error t value
## (Intercept) 251.4051     9.7467   25.79
## Days         10.4673     0.8042   13.02
## 
## Correlation of Fixed Effects:
##      (Intr)
## Days -0.371
\end{verbatim}

\begin{Shaded}
\begin{Highlighting}[]
\KeywordTok{round}\NormalTok{(}\KeywordTok{confint}\NormalTok{(lmm, }\DataTypeTok{oldNames =} \OtherTok{FALSE}\NormalTok{), }\DecValTok{2}\NormalTok{)}
\end{Highlighting}
\end{Shaded}

\begin{verbatim}
##                         2.5 % 97.5 %
## sd_(Intercept)|Subject  26.01  52.94
## sigma                   27.81  34.59
## (Intercept)            231.99 270.82
## Days                     8.89  12.05
\end{verbatim}

\begin{Shaded}
\begin{Highlighting}[]
\NormalTok{lmm_days <-}\StringTok{ }\KeywordTok{lmer}\NormalTok{(Reaction }\OperatorTok{~}\StringTok{ }\NormalTok{Days }\OperatorTok{+}\StringTok{ }\NormalTok{(Days }\OperatorTok{|}\StringTok{ }\NormalTok{Subject), sleepstudy)}
\KeywordTok{summary}\NormalTok{(lmm_days)}
\end{Highlighting}
\end{Shaded}

\begin{verbatim}
## Linear mixed model fit by REML ['lmerMod']
## Formula: Reaction ~ Days + (Days | Subject)
##    Data: sleepstudy
## 
## REML criterion at convergence: 1743.6
## 
## Scaled residuals: 
##     Min      1Q  Median      3Q     Max 
## -3.9536 -0.4634  0.0231  0.4633  5.1793 
## 
## Random effects:
##  Groups   Name        Variance Std.Dev. Corr
##  Subject  (Intercept) 611.90   24.737       
##           Days         35.08    5.923   0.07
##  Residual             654.94   25.592       
## Number of obs: 180, groups:  Subject, 18
## 
## Fixed effects:
##             Estimate Std. Error t value
## (Intercept)  251.405      6.824  36.843
## Days          10.467      1.546   6.771
## 
## Correlation of Fixed Effects:
##      (Intr)
## Days -0.138
\end{verbatim}

\begin{Shaded}
\begin{Highlighting}[]
\KeywordTok{round}\NormalTok{(}\KeywordTok{confint}\NormalTok{(lmm_days, }\DataTypeTok{oldNames =} \OtherTok{FALSE}\NormalTok{), }\DecValTok{2}\NormalTok{)}
\end{Highlighting}
\end{Shaded}

\begin{verbatim}
##                               2.5 % 97.5 %
## sd_(Intercept)|Subject        14.38  37.71
## cor_Days.(Intercept)|Subject  -0.48   0.68
## sd_Days|Subject                3.80   8.75
## sigma                         22.90  28.86
## (Intercept)                  237.68 265.13
## Days                           7.36  13.58
\end{verbatim}

\hypertarget{model-fitting}{%
\subsection{Model Fitting}\label{model-fitting}}

\hypertarget{bayesian-linear-mixed-models}{%
\section{Bayesian Linear Mixed
Models}\label{bayesian-linear-mixed-models}}

A Bayesian version of the LMM is given by the use of prior distributions
for \(\beta\), \(\Sigma\) and \(\Sigma_b\). In the Bayesian case, the
distiction between random and fixed effects is less clear, as both
\(\beta\) and \(b\) will have probability distributions \(f(\beta)\) and
\(f(b) = \int f(b | \Sigma_b) f(\Sigma_b)d\Sigma_b\). The most common
way of fitting the BLMM is based on a Gibbs sampler.

\end{document}
